Ich schreibe dir diesen Brief, weil du dich sicher schon gewundert hast, was
mit mir in letzter Zeit los ist und was mich so stark beschäftigt. Als meine
engste Vertraute hast du ein Recht darauf, zu erfahren, was sich zugetragen
hat. Da du aber ja immer so mitfühlend bist, wollte ich dich nicht gleich in
diese schlimme Sache reinziehen. Indem ich diesen Brief schreibe, versuche ich
dir nun aber alles zu erklären und gleichzeitig meine Gedanken zu ordnen.

Was ich getan habe, ist schrecklich und grenzt an Selbstjustiz. Ich habe heute
beinahe zwei Menschen erschossen. Ich habe am Ostersamstag im Münzloch eine
Leiche gefunden. Dieser Fund beschäftigt mich jetzt schon seit Tagen und je
mehr ich mich damit beschäftige, umso mehr Hinweise bekomme ich, dass diese
Leiche etwas mit mir zu tun hat. Nicht, dass mich der Zustand dieser Leiche
erschreckt hat, nein, eher wie sie zu Tode gekommen ist. Dazu muss ich sagen,
dass die Leiche in einer kleinen Seitenhöhle lag, die absolut luftdicht mit
Lehm abgedichtet war. Als ich den mumifizierten Körper gesehen und genauer
untersucht hatte, weckte das so ein komisches Gefühl in mir. Mir war so, als ob
es meine ureigenste Aufgabe ist, die Umstände aufzuklären, unter denen der
Mensch zu Tode gekommen ist. Ich habe also daraufhin meine eigenen
Nachforschungen angestellt. Ich stellte fest, dass der Mann erschossen wurde.
Es handelte sich hier wahrscheinlich um einen Behinderten, da die Leiche ein
weißes Kreuz auf dem Anzugrücken hat, genauso wie es die Behinderten in
Grafeneck hatten. Die Leiche war völlig sauber. Wie sie aber in die
abgeschlossene Kammer gekommen ist, ist mir bis heute noch schleierhaft. Ich
glaube, dass sie nach dem Mord da hingebracht wurde. Bis ich den Fund der
Polizei gemeldet habe, wartete ich zunächst noch. Ich wusste erst wirklich
nicht, ob ich denen das melden soll. Du kennst ja den Waiblinger, der muss
immer alles gleich an die große Glocke hängen. Bevor ich die Angelegenheit in
fremde Hände gebe, wollte ich selber etwas mehr Klarheit haben. Erst am
folgenden Tag fuhr ich dann zum Waiblinger auf die Wache und habe ihm von der
Leiche erzählt. Er hat jetzt nicht selbst mit dem Fall zu tun, es ist ein
Kommissar aus Reutlingen eingesetzt worden, der sich intensiv damit
auseinandersetzt. Der Kriminalpolizist heißt Greving und ist absolut in
Ordnung. Trotzdem bin ich nochmal auf eigene Faust losgezogen und habe mit
meinem Metalldetektor das Gebiet abgesucht und dabei über der Höhle tatsächlich
eine Patronenhülse aus einer P04 gefunden. Wie du weißt, hatte mein Vater auch
so eine Pistole, ich reinige sie ja jedes Jahr mindestens einmal. Ich frage
mich jetzt natürlich, ob mein Vater irgendetwas mit dem Mord zu tun gehabt
haben könnte. Ich bin mir nicht sicher ob er geschossen hat aber ausschließen
kann ich es leider auch nicht. Diese Ungewissheit über eine Mitwirkung oder gar
Mitschuld meines Vaters frisst mich innerlich auf.

Waltz, von dem ich das mit den weißen Kreuzen weiß, hat mir außerdem von dem
ehemaligen Grafenecker Anstaltsarzt Hochstätter erzählt, der jetzt in
Hundersingen lebt. Zu dem bin ich dann sofort hingefahren und wollte ein
Schuldgeständnis aus ihm rauspressen. Ich bin dabei so weit gegangen, dass ich
ihn sogar mit Vaters alter Waffe bedroht habe. Er wollte mir aber nur noch ein
paar Namen nennen, darunter auch den von Eugen Mattes, der Busse nach Grafeneck
gefahren haben soll, wie Hochstätter mir erzählte. Den habe ich dann sofort
aufgesucht und ihm klargemacht, dass er zumindest eine Mitschuld trägt. Auch
ihn habe ich mit der Waffe bedroht und wollte so sein Schuldeingeständnis
erzwingen.

Mir wird das gerade alles zu viel. Die alten Geschichten ausgraben, aber auch
diese Ungewissheit, das schmerzt mich alles sehr. Ich muss auch ständig an Mutz
denken. Ich weiß nicht, wie emotional ich geworden wäre, wenn ich direkt zu dir
gekommen wäre, nachdem ich Eugen bedroht habe. Ich hoffe du verstehst das. Ich
werde dich wieder besuchen, wenn alles ein wenig ruhiger geworden ist. Ich habe
das Gefühl, dass es meine Verpflichtung, ja mein Schicksal ist, diesen Fall
aufzuklären und was es mit meinem Vater zu tun gehabt haben könnte. Ich kann
mir nicht helfen, da ist irgendwas Geheimnisvolles dabei.
